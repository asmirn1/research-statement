\documentclass[11pt]{amsart}

\usepackage{amsrefs}
\usepackage[top=1in, bottom=1in, left=1in, right=1in]{geometry}
\pagestyle{empty}

\title{Research Statement}

\begin{document}
\begin{center}
	\textsc{Aleksandr Smirnov}
    
    \textsc{Research Statement}
\end{center}

\bigskip

The Wiener--Hopf technique is a powerful analytical tool of solving integral equations and partial differential equations, that has proved to be useful in many problems of physics and engineering from scattering of acoustic or water waves, to viscous and inviscid fluid flows, to crack propagation in elastic media. This technique was developed by Norbert Wiener and Eberhard Hopf in 1931 as a tool to solve the differential equations governing the radiation equilibrium of stars. Nowadays, even more fields of mathematics and physics find use of the technique, including, for instance, analysis of random matrices in mathematical physics and analysis of L\'evy processes in probability theory.

The key part of the Wiener--Hopf technique is factorization of a function, say $a(x)$, on the contour $L$. For simplicity, assume that $L$ is a smooth simple closed curve on the complex plane $\mathbb C$, which splits the complex plane into two  regions $D^+$ and $D^-$ so that $\mathbb C=D^+\cup L\cup D^-$. One needs to find Wiener--Hopf factors, two function $a^+(z)$ and $a^-(z)$ bounded at the infinity, such that $a^\pm(z)$ is analytic and zero--free in the region $D^\pm$, while on the contour $L$ they satisfy the equation
$$
	a(x)=a^+(x)/a^-(x),\quad x\in L
$$
If the function $a(x)$ is such that $\ln a(x)$ is H\"older--continuous on the contour $L$, then the factors $a^\pm(z)$ can be defined by
$$
	a^\pm(z)=\exp\left\{\frac1{2\pi i}\int_{L}\ln a(t)\frac{dt}{t-z}\right\},\quad z\in D^\pm
$$
where the values $a^\pm(x)$ for $x\in L$ are understood as the limits of $a^\pm(z)$ as $z\to x$ while $z\in D^\pm$. The formula above is applicable to a specific class of the functions $a(x)$ but it can be generalized in several ways: (i) relaxing the requirements on the contour $L$; (ii) considering wider classes of the functions $a(x)$; (iii) generalizing the formulas for the case when $a(x)$ is a matrix-function. The latter task seems to be most challenging of the three, and attracts attention of many mathematicians these days. 

In my work so far, I have been dealing with partial differential equations arising from problems of Dynamic fracture mechanics. For each problem, I constructed an analytical or numerical solution with the help of Wiener--Hopf technique with the main focus on deriving an effective solution, which would be easy for numerical computation. The main results of my work are listed below:

The first problem is to study the crack propagation in an elastic media having the shape of a half-space so that the crack is parallel to the boundary of the media. The boundary of the media is tension free, while the the crack faces are subject to an external loading. Under the assumptions of the plane strain deformation, the problem is reduced to a Riemann--Hilbert problem that requires the vector Wiener--Hopf factorization.

The Sokhotski--Plemelj formulas \cites{nobel, gahov} imply that
$$
	a^\pm(\xi)=\exp\left\{\pm\frac12\ln a(\xi)+\frac1{2\pi i}p.v.\int_{-\infty}^\infty\ln a(t)\frac{dt}{t-\xi}\right\},\quad \xi\in\mathbb R
$$
where ``$p.v.\int$'' stands for the principal value of the integral. Thus, the defined above functions $a^\pm(\xi)$ provide the required Wiener--Hopf factorization \eqref{wh_R}. This technique can be applied in order to solve the Riemann--Hilbert problem \cite{gahov} on the real axis $\mathbb R$ and on a variety of other open and closed, smooth and non-smooth curves \cite{gahov}.

The formula \eqref{factorization_R} provide  two functions $a^\pm(\xi)$ that are continuous on the real axis $\mathbb R$ only if $a(\xi)$ has neither poles nor zeros on the integration path, it converges to the same limit as $\xi\to\pm\infty$, and the index of $a(\xi)$, the increment of $\arg a(\xi)$ on $\mathbb R$, is equal to zero. If the function $a(\xi)$ has poles, zeros, or branch points on or near $\mathbb R$, analytical construction and numerical estimation of the Wiener--Hopf factors $a^\pm(\xi)$ may present certain difficulties. Different tricks can be applied to eliminate singularities on $\mathbb R$: the function $a(\xi)$ can be multiplied by a rational function \cite{gahov} of a special form or by a trigonometric function \cite{antipov-willis} that can be factorized explicitly without employing the formula \eqref{factorization_R}; the integration path can be transformed in order to express $a^\pm(\xi)$ through a combination of finite-range integrals or special functions \cite{abrahams}.

This paper is aimed to factorize Wiener--Hopf kernels that are discontinuous on the real axis $\mathbb R$ and have no limit at the infinity. The described method employs changing of an integration path and the Wiener--Hopf factorization on a non-smooth contour in order to deal with lack of convergence and singularities near the integration path. Then, it is applied to a problem of Dynamic fracture mechanics on an intersonic crack propagation in a strip.

Slaipner A, an offshore platform producing oil and gas in the Norwegian sector of the North Sea, is known for its catastrophic failure on August 23 1991, due to a design flaw of the flotation system, resulting in a total loss of the platform. The conclusion of the investigation was that the loss was caused by an inaccurate approximation of the linear elastic model of the construction. The shear stresses were underestimated by 47\%, leading to insufficient design. In particular, certain concrete walls were not thick enough.

Many well known stories of the sort give a big motivation to study Dynamic Fracture Mechanics. How cracks form and propagate in a material? What can we say about minimum and maximum speed of a crack? Under what conditions a crack starts branching? How the crack characteristics affect the material stiffness and its resistance to failure? In order to answer these questions, numerous experiments were performed, numerical models constructed, and analytical solutions derived. My special interest here is in analytical solutions of Dynamic Fracture Mechanics problems. Those solutions, while often being elaborated and challenging, provide an important understanding of crack propagation phenomena. So far, I have solved several problems of a crack propagation in materials of different types and shapes and published one and submitted two papers with the results of my research.

In the conventional elastic Fracture Mechanics, cracks are modeled as discontinuities of the stress $\boldsymbol\sigma$ and deformation $\boldsymbol\varepsilon$ fields in the body. Many characteristics such as direction, speed, and feasibility of a crack propagation are governed by the stress field near the crack tip and determined based on so-called stress intensity factors. Three regimes of propagation are typically considered. In the subsonic regime, speed of the crack propagation $v$ is limited above by the Rayleigh speed $c_R$ of the material. For a mode I crack, the energy flux into the crack tip region vanishes at $c_R$ and no analytical solution can be found for higher speeds \cite{broberg} (in the framework of linear elasticity). The intersonic regime assumes the crack propagation speed $v$ in the range between the shear wave speed $c_s$ and the longitudinal wave speed $c_l$ of the material. It has been studied for mode II cracks \cites{andrews,burridge} and has attracted much attention in the last decade due to Rosakis' experimental work \cite{rosakis}. Gao et al. \cite{gao} demonstrated existence of intersonic cracks under shear dominated conditions from the atomic point of view. The works \cites{abraham,buehler} are devoted to the supersonic regime ($v>c_l$) and utilize non-linear theory of hyperelasticity to justify the crack propagation. For a detailed review of different aspects of dynamic fracture mechanics, see \cite{cox}.
    
    For a mode II crack, the subsonic and intersonic regimes are admissible, while the regime $c_R<v<c_s$ is forbidden. However, the stress field near the crack tip has a power singularity of order $1/2$ (in the case $v=\sqrt2c_s$) or less, resulting in zero energy release rate for all intersonic propagation speeds $v\neq\sqrt2c_s$ \cite{freund}. To overcome the difficulty, a cohesive zone of finite length in front of the crack tip is assumed, consisting of upper and lower surfaces held by the cohesive traction. The model was originally proposed by Barenblatt in \cite{barenblatt}. The advance of the crack is viewed as the breaking of the atomic bonds between pairs of atoms, while bonding force (the cohesive traction) is related to the separation of the opposite faces of the crack according to a cohesive law. The separation is assumed to be zero at the mathematical tip of the crack (beginning of the cohesive zone), it increases behind the mathematical tip and reaches a critical value at the end of the cohesive zone, which is a physical tip of the crack. Thus, a point singularity at the crack tip is replaced by a zone of finite length with a non-zero energy flux into the zone. The fundamental solution for intersonic mode II cracks is presented in \cite{huang} and utilizes the cohesive zone model satisfying the Dugdale cohesive law \cite{dugdale}. In the case when an external load is negligible compared to the cohesive traction, the solution in a simple closed form is found in \cite{antipov}. The latest advances in the cohesive zone approach include fracture in homogeneous media under mixed mode conditions \cites{ortiz,xu} and along bi-material interfaces \cites{needleman,tvergaard,tvergaard2001}.
    
    This paper is a result of the author's efforts in the area of dynamic fracture mechanics, started in the work \cite{antipov2012}. It features a new (to the author's knowledge) effective Wiener--Hopf factorization of the coefficient of the Riemann-Hilbert boundary value problem \cite{gakhov}, having infinitely many zeros and poles on the boundary. In Section \ref{sec:derivation}, the problem of an intersonic steady-state crack propagation along an interface on the symmetric axis of an infinite strip with crack faces subjected to an arbitrary shear traction is reduced to a boundary value problem on a complex plane. Section \ref{sec:solution} develops an analytic solution of the Riemann-Hilbert problem. Using the cohesive zone model with the Dugdale cohesive law, the solution of the original physical problem is derived in Section \ref{sec:cohesive_model}, and dependence of the cohesive zone length and the energy release rate versus width of the strip is described. The appendix sections contain derivation of some formulas used for numerical estimation of the energy release rate.

I came to the field of Fracture Mechanics as an undergraduate student of Department of Mathematics (Chuvash State University, \emph{Russia}), working with my adviser Prof. V. Silvestrov on the problem of strengthening of material interface with a composite stringer. This work put me into the field of the conventional theory of linear elasticity and analytical methods of Complex Analysis. During the years at Louisiana State University, I have significantly improved my understanding of Fracture Mechanics and have gained a unique experience working with Dr. Y. Antipov. The result of these years is three papers written by me and Dr. Y. Antipov as a co-author. In the first two works, me and Dr. Y. Antipov address the problem of a crack propagating in a half-plane under assumptions of the steady-state (when the stress intensity factors $K_\alpha$ are unchanging in time) and the transient (when $K_\alpha$ are changing in time) motion. Essential for those works was to consider a crack in an inhomogeneous half-plane, which wasn't done before for the half-plane, and to construct an analog of the Griffith criterion of crack propagation for the half-plane. On my part, I derived analytical and numerical solution of the problem (with a vital help of Dr. Antipov), and wrote a computer program in Fortran to perform evaluations.

My third paper is about a crack in a symmetric strip, propagating with an intersonic speed. The entire work was performed by me with a few valuable comments and tips from my adviser. Unexpectedly, while working on the problem, I faced inability of the linear elastic fracture mechanics to describe some phenomena of crack motion. In order to consider the intersonic speed of crack propagation, I implemented the cohesive zone model, which allowed me to obtain the analytic solution for the problem, to find length of the cohesive zone and the fracture energy release rate, and to analyze their dependence on the crack speed and width of the strip. The next logical step would be to consider an intersonic crack in an inhomogeneous strip under the cohesive zone assumptions, but Dr. R. Lipton suggested for me to consider the lattice model, which I'm currently working on.

\bigskip

I am an applied mathematician mostly interested in solving problems of Dynamic Fracture Mechanics. So far my research has been focused on using analytical tools of Complex Analysis as well as numerical methods to describe behavior of a crack propagating in various materials. For me, it is fascinating and challenging opportunity to apply my many years' experience in the area of Complex Analysis devoted to solving the Riemann--Hilbert problems (widely used in Fracture Mechanics), and in writing software on several programming languages.

\bigskip

Nowadays, the process of designing various machine components consists of choosing an appropriate geometry, necessary material strength, working temperature range, and other numerous criteria in order to provide safety and effectiveness of developed devices. Fracture mechanics is one of the main tools in improving mechanical properties of components. It is concerned with the study of the propagation of cracks in materials. The mathematical solutions of Fracture Mechanics problems are often extremely elaborated and challenging, while they provide an important understanding of processes intrinsic to crack motion. 

\bigskip

\textbf{Dynamic Fracture Mechanics.} One of the first thorough studies of a crack in a linearly elastic material was conducted by G.R. Irwin in 1958. He first noted that the stress field $\sigma_{ij}$ near the crack tip has a square root singularity in the form
$$
	\sigma_{ij}=\frac{K_\alpha}{\sqrt{2\pi r}}f^\alpha_{ij}(v,\theta)
$$
where $(r,\theta)$ are polar coordinates, $v$ is the instantaneous crack velocity, $\alpha$ is an index corresponding to three different ways of loading, and $f_{ij}^\alpha(v,\theta)$ is a known universal function. The coefficient $K_\alpha$, called \emph{stress intensity factor,} contains all of the information about sample loading and history, and entirely determines the behavior of a crack. Thus, much of the study of fracture consists of calculation or measurement of $K_\alpha$ in the framework of the linear elasticity.

How do results of linear elastic fracture mechanics compare with experiments? The theory has been highly successful in predicting the values of $K_\alpha$ for stationary and dynamic cracks in a broad range of materials and their shapes. However, as was first noted in the work by A. Kobayashi et al. (1974), the theory fails to predict such parameters of a crack propagation as the limiting velocity and stability of a crack in brittle materials. According to the linear elastic fracture mechanics, the stress fields $\sigma_{ij}$ diverge as $\left.1\middle/\sqrt{r}\right.$ approaching the crack tip. In reality, stress cannot diverge. Therefore, there exists a small process zone near the crack tip in which the actual phenomena are different from those predicted by linear elasticity. It is believed, in particular by J. Fineberg and M. Marder, that the reasons for the discrepancy between the theory and the experiment on brittle materials are due to the growth of the process zone to a scale sufficient to violate the predictions of the linear elastic fracture mechanics. Numerous models were proposed to explain the discrepancy. The cohesive zone of Barentblatt and Dugdale is the simplest model of the process zone. Assume that from a tip of the crack, a uniform stress $\sigma_c$ acts between the stress surfaces and drops to zero at the point where the surfaces separation reaches a critical value $s$. The idea of the cohesive zone model is to choose $s$ and $\sigma_c$ in order to eliminate the square root singularity. Cohesive zones of this kind are frequently observed in the fracture of polymers. 

However, there are many others phenomena unexplained by the conventional fracture mechanics: patterns on the fracture surface, roughness of the fracture surface, microscopic and macroscopic crack branching, etc. In the last two decades, many of them were addressed with the lattice model of brittle materials (ceramics, glass, PMMA also known as Plexiglas, etc.) introduced by L. Slepyan in 1981 and by M. Marder and X. Liu in 1993. The model makes use of arranging atoms in a crystal and assuming a simple force law between them. This force law makes it possible to obtain a large variety of analytical results for fracture in arbitrary large systems. It was used by S.A. Kulakhmetova to find exact analytical expressions for cracks moving in a square lattice, by R. Thomson to describe lattice trapping of fracture cracks, by M. Marder and S.P. Gross to show existence of a minimum allowed crack velocities, to analyze branching instabilities of cracks, and to estimate spacing between branches. Several works by L. Slepyan and G.S. Mishuris were intended to consider lattice fracture in inhomogeneous materials. The lattice model is now one of the most rapidly developing ideas in Fracture Mechanics and attracts attention of many researchers. That is why my current research is based on the lattice model and on brilliant works by such authors as M. Marder and S.P. Gross.

\bigskip

\textbf{Past research.} I came to the field of Fracture Mechanics as an undergraduate student of Department of Mathematics (Chuvash State University, \emph{Russia}), working with my adviser Dr. V. Silvestrov. During the first three years at LSU, I have significantly improved my understanding of Fracture Mechanics and have gained a unique experience working with Dr. Y. Antipov. The result of these years is three papers written by me and Dr. Y. Antipov as a co-author. In the first two works, me and Dr. Y. Antipov address the problem of a crack propagating in a half-plane under assumptions of the steady-state (when the stress intensity factors $K_\alpha$ are unchanging in time) and the transient (when $K_\alpha$ are changing in time) motion. Essential for those works was to consider a crack in an inhomogeneous half-plane, which wasn't done before for the half-plane, and to construct an analog of the Griffith criterion of crack propagation for the half-plane. On my part, I derived analytical and numerical solution of the problem (with a vital help of Dr. Antipov), and wrote a computer program in Fortran to perform evaluations.

My third paper is about a crack in a symmetric strip, propagating with an intersonic speed. The entire work was performed by me with a few valuable comments and tips from my adviser. Unexpectedly, while working on the problem, I faced inability of the linear elastic fracture mechanics to describe some phenomena of crack motion. In order to consider the intersonic speed of crack propagation, I implemented the cohesive zone model, which allowed me to obtain the analytic solution for the problem, to find length of the cohesive zone and the fracture energy release rate, and to analyze their dependence on the crack speed and width of the strip. The next logical step would be to consider an intersonic crack in an inhomogeneous strip under the cohesive zone assumptions, but Dr. R. Lipton suggested for me to consider the lattice model, which I'm currently working on.

\bigskip

\textbf{Current and Future research.} Inspired by the work of M. Marder and S.P. Gross (1995) on instabilities of a crack in brittle materials, which effectively utilizes the lattice model for a material, I have chosen to apply the lattice model to a problem of a crack propagation in an inhomogeneous strip. Moreover, I plan to extend their work in several directions:
\begin{enumerate}
	\item considering an inhomogeneous strip (to the best of my knowledge, only a few works on the lattice model analyze fracture of inhomogeneous material);
    \item considering the so-called mixed mode fracture when the crack face displacements are allowed to occur in different directions (it is conventional to consider only one direction since derived solutions are much simpler in this case, but the inhomogeneity of the strip requires to consider the mixed-mode fracture);
    \item considering a crack propagation at an intersonic speed (the phenomenon of a crack motion at extremely high speed was discovered at the very end of 90s and is not studied well enough yet.)
\end{enumerate}

\bigskip

Funding of my research will help me to finish it within one year (since currently a lot of my attention is devoted to my teaching assistantship duties). If funded, the timeline for completion of my dissertation will look as follows: deriving the analytic and numerical solutions of the problem on a crack propagating in an inhomogeneous strip under anti-plane loading (Spring, 2015); analyzing numerical results and deriving dependence of crack velocity and stability on parameters of a material (Summer, 2015), possible collaboration with engineers and comparing the solution to experiments, and/or deriving solution for the problem under mixed-mode loading (Fall, 2015), and writing a dissertation thesis (Spring, 2016).

\end{document}