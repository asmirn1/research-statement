\documentclass[11pt]{amsart}

\usepackage{amsrefs}
\usepackage[top=1in, bottom=1in, left=1in, right=1in]{geometry}
\pagestyle{empty}

\title{Research Statement}

\begin{document}
\begin{center}
	\textsc{Aleksandr Smirnov}
    
    \textsc{Research Statement}
\end{center}

\bigskip

In complex analysis, the Riemann--Hilbert problem is a problem of determining analytic functions satisfying to a boundary condition of a special form. It gave rise to a powerful method of solving integral equations and partial differential equations, that has proved to be useful in many problems of physics and engineering from scattering of acoustic or water waves, to viscous and inviscid fluid flows, to crack propagation in elastic media. The Riemann--Hilbert problem is genuinely related to the Wiener--Hopf technique of the analytic factorization of a given function defined on a curve, which was first studied This technique was developed by Norbert Wiener and Eberhard Hopf in 1931 as a tool to solve the differential equations governing the radiation equilibrium of stars. Nowadays, even more fields of mathematics and physics find use of the method, including, for instance, analysis of random matrices in mathematical physics and analysis of L\'evy processes in probability theory.

For simplicity, assume that $L$ is a smooth simple closed curve on the complex plane $\mathbb C$, which splits the complex plane into two  regions $D^+$ and $D^-$ so that $\mathbb C=D^+\cup L\cup D^-$. The Riemann--Hilbert problem consists of determining two functions $\phi^\pm(x)$ such that they are analytic and zero-free in $D^\pm$ respectively, bounded at the infinity, and satisfy the boundary condition
$$
	\phi^+(x)=G(x)\phi^-(x)+g(x),\quad x\in L
$$
The key step in its solution is the Wiener--Hopf factorization of the function $G(x)$, which is a solution of the Riemann--Hilbert problem with the homogeneous boundary condition (that is, $g(x)\equiv 0$). If $\ln G(x)$ is H\"older--continuous on the contour $L$, such a solution can be given by
$$
	\phi_0^\pm(z)=\exp\left\{\frac1{2\pi i}\int_{L}\ln G(t)\frac{dt}{t-z}\right\},\quad z\in D^\pm
$$
where the values $\phi^\pm(x)$ for $x\in L$ are understood as the limits of $\phi^\pm(z)$ as $z\to x$ while $z\in D^\pm$. The formula above is applicable to a specific class of the functions $G(x)$ but it can be generalized in several ways including: (i) relaxing the requirements on the contour $L$; (ii) considering wider classes for the function $G(x)$; (iii) generalizing the formulas for the case when $G(x)$ is a matrix-function. The latter task seems to be most challenging of the three: a closed-form solution was constructed only for a small class of so-called Chebotarev--Khrapkov matrices, while other matrix-functions $G(x)$ admit only approximate factorization.

In my research so far, I have been dealing with several partial differential equations arising from problems of Dynamic fracture mechanics. Each equation was reduced to a boundary condition of the Riemann--Hilbert problem and their analytical or numerical solutions were constructed with emphasis on deriving an effective solution, which would be easy for numerical computation. The main parts of my research are listed below:

In the first part of my research, I studied the crack propagation in an elastic media having the shape of a half-space so that the crack is parallel to the boundary of the media. The boundary of the media is tension free, while the the crack faces are subject to an external loading. Under the assumptions of the plane strain deformation, the problem is reduced to a vector Riemann--Hilbert problem with $2\times2$ matrix-function $G(x)$, which does not admit a closed--form factorization. Several approximate solutions were proposed and constructed:
\begin{enumerate}
	\item The Riemann--Hilbert problem was reduced to a singular Fredholm integral equation of the second kind and solved approximately by mean of the Galerkin method. An orthonormal set of Jacobi polynomials was chosen in order to deal with singularities of $G(x)$ at the end-points of the contour $L$.
    \item In the case of the dynamic problem of the crack propagation in a half-space, the convergence rate of the Galerkin method was found insufficient. In order to improve numerical calculations, the Wiener--Hopf factorization was implemented. Although, the matrix $G(x)$ cannot be factorized explicitly, its approximate factorization was derived as follows: the diagonal elements of the matrix--function $G(x)$ were factorized, using the standard Wiener--Hopf technique, which admitted reducing the Riemann--Hilbert problem to integral equations with non-singular kernels. The collocation method was used to solve the integral equations.
    \item A closed--form solution of the vector Riemann--Hilbert was also derived by solving the corresponding scalar Riemann--Hilbert problem on the Riemann surface. This technique was successfully implemented in the recent works of Y. Antipov and V. Silvestrov. In the case of the problem in question, solution of the Riemann--Hilbert problem on the Riemann surfaces of the infinite genus was constructed, but it was found to be time--inefficient.
\end{enumerate}
As a result of my work, one paper is published and two are submitted to publication with the described above solutions. I have learned several methods of solving partial differential equations: from different numerical methods (for instance, the Galerkin method and the method of collocation points), to an effective use of the Wiener--Hopf technique (including various tricks to deal with the kernel singularities). I gained a great experience in solving problems of Dynamic fracture mechanics in the frameworks of the continuous elastic model of a crack propagation, the cohesive zone model of a crack propagation on intersonic and supersonic speeds, and the lattice model of a crack propagation. I improved my programming skills in FORTRAN and Mathematica (including parallel computations).

But my research interests are not limited by the Riemann--Hilbert problem and Dynamic fracture mechanics. In the last two years, there was an additional source for my scientific interests: teaching an undergraduate course, so-called Capstone course, aimed to allow students to apply their mathematical knowledge to the real--world problems. Working in a small group of researchers and applying mathematics to a wide range of problems was extremely beneficial for me as well. The first problem we worked on was about determining of 2D--locomotion of a robot based on the data received by its optical sensors. In theory, two sensors can fully determine the robot's path on a surface. However, the majority of our efforts were directed on eliminating the data errors due to low accuracy of the optical sensors, inconsistency of geometrical position of the robot's sensors, and different optical properties of the surface. However, a robot with eight optical sensors was implemented and the corresponding software was developed in MATLAB. The second problem we worked on was about analyzing data from eye-tracking equipment. We analyzed data from the experiment held on undergraduate students and professional pilots of registering their eye-movements during their search of a target on a map. This can be described as a Bid Data problem: find characteristics of eye-movement and eye-movement patterns distinguishing experts from novices, based on the results of more then a thousand recorded trials. In order to achieve the objective we learned several techniques used in Machine learning including Naive Bayes classifier, Information Gain analysis, artificial neural networks, and several methods to measure distances between the eye-paths. We worked on several programming languages (MATLAB, R-language, and Python) and created software that helps non-mathematicians in analyzing characteristics and patterns of the eye-movements.

\end{document}